%----------------------------------------------------------------------------
\chapter{Tervezés és implementáció}
%----------------------------------------------------------------------------

\section{Tervezés, döntési lehetőségek értékelése}

\subsection{Python}

A Python napjaink egyik legelterjedtebb programozási nyelve. Egy könnyen programozható és átlátható, interpretált nyelv, melyben a megírt program kódot a Python értelmező sorról-sorra értelmezi és futtatja, nincs különválasztva a forrás és a fordított kód. Deep learning körökben is a Python nyelv dominál, rengeteg Deep learning toolkit erre a nyelvre épít, így választásom a fejlesztéshez természetesen a Python nyelvre esett.

\subsection{NeMo}

A neurális hálók komplikáltságából adódik, hogy pontos és precíz implementációjuk igen bonyolult és időigényes. Emiatt célszerű a már meglévő, bejáratott és bizonyított lehetőségeket használni, így felgyorsítva a fejlesztést. Több deep learing toolkit is található az internetet, mint például a Tensorflow vagy PyTorch. Az Nvidia által fejlesztett NeMo toolkit,

\subsubsection{PyTorch}

\subsubsection{QuartzNet}

\subsection{Tensorboard}

\section{NeMo beállítása}

\section{Kísérletek}