%----------------------------------------------------------------------------
\chapter{Feladatkiírás részletes elemzése}
%----------------------------------------------------------------------------

\section{Beszédfelimserés bemutatása}

Az irodalomkutatás során az end-to-end alapú beszédfelismerést különböző források, cikkek, internetes bologok és papírok alapján célszerű végezni. Mivel egy egészen új és feltörekvő megközelítésről van szó, célszerű a forrásokat megvizsgálni, összevetni az azonos témáról írtakat az alátámasztás és a könnyebb értelmezhetőség végett.

A neurális hálók prezentálása elengedhetelen, lényegretörő bemutatásuk és szerepük nyomán a beszédfelimserésben. Fontos kitérni az end-to-end alap mivoltára, ennek előnyeinek és hátrányainak bemutatása az eddig bevált, hagyományos módszerekkel szemben.

Érdemes az elméleten túl a potenciálisan felhasználható megoldásokat részletesen megismerni, melyek implementálhatók a későbbi tervezés során a jobb eredmények végett.

\section{Korszerű architektúrák és tervezés}

Többféle neuronhálós architketúrák léteznek különböző feladatokra, így a beszédfelismerésre is. Felderítésük, előnyeik és hátrányaik megismerése és összevetésük az irodalomkutatás nyomán végezhető el. 

A neurális hálók természetüknél fogva igen összetettek, ezért gyakori az egyes rendszerek magas erőforrásigénye, melyeket a diplomamunka keretein belül akár kivitelezhetetlen is lenne használni. Ezért célszerű, hogy olyan architektúrát válasszunk, ami az előzőek alapján optimális teljesítményt nyújt, azaz látványos és felhasználható eredmények elérésére képes a lehető legrövidebb idő alatt a legkevesebb erőforrást felhasználva. Az idő rövidsége a tesztelési fázist is nagyban felgyorsíthatja, így szerepe kulcsfontosságú, míg természetesen fontos és nem elhanyagolható a pontosság is.

Kezdetben angol nyelven célszerű tesztelni az megtervezett architektúrát és az egyes paramétereinek pontosítását, mivel angol nyelven rengeteg eredmény elérhető, ezáltal összevethető. Ha egy megtervezett rendszer működésének megítélése angol nyelvű adatok alapján megfelelő akkor már célszerű áttérni az orosz nyelvre. 

\section{Orosz nyelvű implementáció}

Egy angoltól eltérő nyelvre való áttéréskor el kell végezni annak a vizsgálatát, hogy az újabb nyelv milyen különbségeket rejt. Egy gyakori eltérés az ABCi, de előfordulhatnák más előfeldolgozásbeli   különbségek is. Ha egy keretrendszer angol nyelvre van optimalizálva, előfordulhat, hogy ezeket a finomításokat el kell vetni és egy általános megközelítést kell alkalmazni az új nyelv tanítása során.

A célnyelvi adatokat különböző forrásokból lehet begyűjetni. Mivel az adatbázisok csak nyíltak lehetnek, ezért ezek megbízhatóságát meg kell vizsgálni akár szúrópróbaszerűen akár más felhasználók véleménye alapján. Az adatok mérete mind a neurális modellek tanítási ideje mind tárhelykapacitás miatt nem szabad, hogy túl nagy legyen, meg kell találni az optimális mennyiséget.

Tanítás során a jobb eredmény elérése érdekében alkalmazható a transfer-learning, ami egy elterjedt tanítási módszer neurális hálók használata során. Ez a folyamat egy adott architektúrájú, előre tanított modell továbbtanítását foglalja magába. Beszédfelismerés esetén az eredeti modelltől eltérő nyelv alkalmazásánál kihívást jelent a nyelvre való áttérés. Szerencsére lehetőség van az előre tanított modell tetszőleges részeit átvenni, nem szükséges a nyelvspecifikus elmeket is importálni, így azok tanításhoz kezdődhet nulláról.

Célszerű az architektúra tervezésénél gondolni a szabadon elérhető előre tanított modellekre és ilyen felépítéseket is fontolóra venni és megvizsgálni, hiszen csak ezekkel lehetséges a transfer-learning alkalmazása.

\section{Optimaliziálás és kiértékelés}

Az elkészült modellt ki kell értékelni, összehasonlítani más eredményekkel és vélemenyezni az így elért eredményt. A kiérétkeléshez egy elterjdet módszert kell ezért használni, hogy összemérhetőek elgyenek az eredmények.

Lehetőség van az eredmény javítására különböző módszerekkel, melyek már ismertek a neurális hálók körében, vagy főként beszédfelismerésnél jelennek meg. Az előbbi csoportba tartozik például az augmentálás, mely folyamat során a meglévő tanítóhalmazt dúsítják különböző módszerekkel. Utóbbi csoportba tartoznak a különböző dekóderek, annak a módja hogy hogyan értelmezzük a valószínűségként kapott kimeneteket az egyes időpillanatokban, illetve a nyelv modellek.

\section{Munka összegzése és kitekintés}

Befejezeséképp kerül sor a kitűzött fealdatok sikerességének bemutatására, az elért eredmények részletezéséve és az esetleges kudarcok a zsákutcák leírására.

További, ki nem próbált lehetőséget és fejlesztési ötletek javaslását is elvégezzük, indokolva, hogy miért is jelenthetnek fejlődést az eredményekben és hogy ezek miért nem lettek megvalósítva, kipróbálva.