\pagenumbering{roman}
\setcounter{page}{1}

\selecthungarian

%----------------------------------------------------------------------------
% Abstract in Hungarian
%----------------------------------------------------------------------------
\chapter*{Kivonat}\addcontentsline{toc}{chapter}{Kivonat}

Amióta lehetséges az emberi hang digitális rögzítése, roppantul izgatja fantáziánkat annak automatikus értelmezése, írásos formában való rögzítése. Ennek számos előnye van, többek közt egy gyors kommunikációs lehetőséget jelent a géppel, ami speciális helyzetekben igen hasznos tud lenni, mint például mentő hívása, de a mindennapokban is jelentős kényelmi szempontot jelent.

Számos implementáció létezik, mint az Amazon Echo, vagy a Microsoft Cortana személyi asszisztense, melyek a bemenő hangot értelmezve igyekeznek válaszolni kérdéseinkre vagy egyéb feladatok elvégzését végezhetik. Ezek a termékek és alkalmazások először az angol nyelvterületeken terjedtek el, aminek oka a nyelvben magában keresendő, hiszen kezdetben angol nyelvű társalgásra, bemenetre készültek, és még a mai napon csak egy maroknyi nyelven lehet kommunikálni vele. A világon rengeteg nyelv létezik és többnél is felmerül az igény, hogy létezzen hozzájuk egy működő beszédfelismerő alkalmazás. A fentiek miatt munkám során az egyik fontos, de a beszédfelismerés szempontjából elhanyagoltabb nyelvvel, az orosszal foglalkoztam. 

Egy fontos eltérés a munkámban hagyományos megközelítésekhez képest, hogy neurális hálókat használ, csökkentve ezzel a felismeréshez használt elemek számát. A feladat során megvizsgáltam annak lehetőséget, hogy egy létező, orosztól eltérő nyelvmodellből kiindulva lehetséges-e jobb eredményeket elérni, mint a semmiből felépítve, és igen biztató eredményeket értem el. A jövőben több módszert is lehet alkalmazni, vagy a meglévő paramétereket optimalizálni a dekódolás pontosságának javítása végett.

\vfill
\selectenglish


%----------------------------------------------------------------------------
% Abstract in English
%----------------------------------------------------------------------------
\chapter*{Abstract}\addcontentsline{toc}{chapter}{Abstract}

This document is a \LaTeX-based skeleton for BSc/MSc~theses of students at the Electrical Engineering and Informatics Faculty, Budapest University of Technology and Economics. The usage of this skeleton is optional. It has been tested with the \emph{TeXLive} \TeX~implementation, and it requires the PDF-\LaTeX~compiler.


\vfill
\selectthesislanguage

\newcounter{romanPage}
\setcounter{romanPage}{\value{page}}
\stepcounter{romanPage}