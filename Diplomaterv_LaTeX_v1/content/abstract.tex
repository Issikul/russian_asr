\pagenumbering{roman}
\setcounter{page}{1}

\selecthungarian

%----------------------------------------------------------------------------
% Abstract in Hungarian
%----------------------------------------------------------------------------
\chapter*{Kivonat}\addcontentsline{toc}{chapter}{Kivonat}

Napjainkban egyre inkább teret nyerő technológia az automatikus beszédfelismerés (ASR) területén a neurális hálók alkalmazása. Az új típusú hálók kutatásával és keretrendszerek fejlesztésével egyre könnyebben és gyorsabban lehet elérni jobb, pontosabb felismerési eredményeket.

A létező implementációk jelentős része a nagyméretű tanító adathalmazzal rendelkező angol nyelv felismerésére lett megvalósítva. Munkám során egy másik fontos, de kevésbé kutatott és csekélyebb mennyiségű tanító adattal rendelkező nyelvre, az oroszra összpontosítok. Több orosz nyelvű, szabadon hozzáférhető adatbázist is megvizsgálok azok kiterjedése és hanganyaguk minősége alapján.

A megvalósítás során az alacsony paraméterszámú, ezáltal kevésbé erőforrásigényes QuartzNet architektúrával dolgozom, az Nvidia NeMo toolkit-jét felhasználva. Röviden felvázolom a NeMo használatához szükséges függőségeket és ismereteket, az adatok szükséges előfeldolgozásának lépéseivel. Külön megvizsgálom a transfer learning nyújtotta lehetőségeket, mellyel sikerült javítsam az elért a felismerés pontosságát, a szó-hiba-arányt (WER).

\vfill
\selectenglish


%----------------------------------------------------------------------------
% Abstract in English
%----------------------------------------------------------------------------
\chapter*{Abstract}\addcontentsline{toc}{chapter}{Abstract}

This document is a \LaTeX-based skeleton for BSc/MSc~theses of students at the Electrical Engineering and Informatics Faculty, Budapest University of Technology and Economics. The usage of this skeleton is optional. It has been tested with the \emph{TeXLive} \TeX~implementation, and it requires the PDF-\LaTeX~compiler.


\vfill
\selectthesislanguage

\newcounter{romanPage}
\setcounter{romanPage}{\value{page}}
\stepcounter{romanPage}