\pagenumbering{roman}
\setcounter{page}{1}

\selecthungarian

%----------------------------------------------------------------------------
% Abstract in Hungarian
%----------------------------------------------------------------------------
\chapter*{Kivonat}\addcontentsline{toc}{chapter}{Kivonat}

Napjainkban egyre inkább teret nyerő technológia az automatikus beszédfelismerés (ASR) területén a neurális hálók alkalmazása. Az új típusú hálók kutatásával és keretrendszerek fejlesztésével egyre könnyebben és gyorsabban lehet elérni jobb, pontosabb felismerési eredményeket.

A létező implementációk jelentős része a nagyméretű tanító adathalmazzal rendelkező angol nyelv felismerésére lett megvalósítva. Munkám során egy másik fontos, de kevésbé kutatott és csekélyebb mennyiségű tanító adattal rendelkező nyelvre, az oroszra összpontosítok. Több orosz nyelvű, szabadon hozzáférhető adatbázist is megvizsgálok azok kiterjedése és hanganyaguk minősége alapján.

A megvalósítás során az alacsony paraméterszámú, ezáltal kevésbé erőforrás-igényes QuartzNet architektúrával dolgozom, az Nvidia NeMo toolkit-jét felhasználva. Röviden felvázolom a NeMo használatához szükséges függőségeket és ismereteket, az adatok szükséges előfeldolgozásának lépéseivel. Külön megvizsgálom a transfer learning nyújtotta lehetőségeket, mellyel sikerült javítanom az elért a felismerés pontosságán, a szó-hiba-arányon (WER).

\vfill
\selectenglish


%----------------------------------------------------------------------------
% Abstract in English
%----------------------------------------------------------------------------
\chapter*{Abstract}\addcontentsline{toc}{chapter}{Abstract}

Nowadays, neural networks are an increasingly widespread technology in the field of automatic speech recognition (ASR). With the research of new types of networks and the development of new frameworks it is easier and faster to produce better, more accurate recognition results than ever before.

Much of the existing implementations have been created to recognize English, which has a considerably large teaching data set. In my work, I focus on an other important, but less researched language, with a more limited training data set, Russian. I also examine multiple freely accessible Russian datasets, based on their lengths and the quality of the recorded audio files.

During the implementation I work with Nvidia's NeMo toolkit and the QuartzNet architecture which is know for its low number parameters, and so it is less resource heavy. I will briefly outline the dependencies and knowledge required to use NeMo, including the steps required to pre-process the collected data. I separately examine the possibilities provided by transfer learning, with which I managed to improve the accuracy, the word-error rate, of the recognition.


\vfill
\selectthesislanguage

\newcounter{romanPage}
\setcounter{romanPage}{\value{page}}
\stepcounter{romanPage}